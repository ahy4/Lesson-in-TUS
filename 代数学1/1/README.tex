\documentclass[a4j,12pt]{jreport}
\title{ {\LaTeX} 動作確認テスト・サンプルファイル}
\author{情報リテラシTA}
\date{\today}
\begin{document}
\maketitle


\def\tightlist{\itemsep1pt\parskip0pt\parsep0pt}
\section{0章 素朴集合}\label{ux7ae0-ux7d20ux6734ux96c6ux5408}

\subsection{定義 0.1}\label{ux5b9aux7fa9-0.1}

集合AからBへの写像fに対して、 1.
\(\forall{} a, b \in A f(a) = f(b) \Rightarrow{} a = b\) を満たす時、fは単射と呼ばれる
2. \(\forall{} c \in B, \exists a f(a) = c\) を満たす時fは全射 3.
単射かつ全射は全単射(bijection)

\subsection{定義 0.2}\label{ux5b9aux7fa9-0.2}

集合A,Bに対して
\(A〜B : AからBへの全単射が存在する\) で〜を定義する(〜は同値関係)
すなわち、 1. A〜A 2. A〜B-\textgreater{}B〜A 3.
A〜BかつB〜C-\textgreater{}A〜C 一般にAの同値関係〜に寄る同値種とは、
\([A]_〜\) = B〜AなるBの集まり のこと

\subsection{定義 0.3}\label{ux5b9aux7fa9-0.3}

集合Aに対して 定義02で定めた同値関係〜に対する同値種\([A]_〜\)を
\textbar{}A\textbar{}と表し、 Aの濃度(cordinality)と呼ぶ。
※Aが有限集合の場合、例えばA=\{a, b\} \textbar{}A\textbar{} =
\textbar{}\{ 0,1 \}\textbar{} = \textbar{}\{1, 2\}\textbar{} = \ldots{}
\textbar{}A\textbar{}を普通は2と表す Aが有限集合の場合は
\textbar{}A\textbar{}をAのこすうの自然数と同一視する

\subsubsection{なぜ同値類は同値集合ではないのか?}\label{ux306aux305cux540cux5024ux985eux306fux540cux5024ux96c6ux5408ux3067ux306fux306aux3044ux306eux304b}

一般に、同値類(class)は集合にはならない。

\subsection{定義 0.4}\label{ux5b9aux7fa9-0.4}

\textbar{}A\textbar{}と\textbar{}B\textbar{}に対して、AからBへの単射が存在する時、
$\textbar{}A\textbar{} \leq \textbar{}B\textbar{} $と表す

\textbar{}A\textbar{}=\textbar{}C\textbar{},
\textbar{}B\textbar{}=\textbar{}D\textbar{}で、
AからBへの単射が存在する時、 CからDへの単射が存在する必要がある。
この定義はwell-defined

\subsubsection{well-defined 補足}\label{well-defined-ux88dcux8db3}

\(frac{b}{a} \oplus frac{d}{c} = frac{b+d}{a+c}\) はwell-definedではない
\(frac{1}{2} \oplus frac{1}{3} = frac{2}{5}\)
\(frac{2}{4} \oplus frac{1}{3} = frac{3}{7}\)

\subsection{定理 0.1}\label{ux5b9aux7406-0.1}

\begin{enumerate}
\def\labelenumi{\arabic{enumi}.}
\tightlist
\item
  \(|A| \leq |B| かつ |B| \leq |C| \Rightarrow |A| \leq |C|\)
\item
  \(|A| \leq |B| かつ |B| \leq |A| \Rightarrow |A| = |C|\)
\end{enumerate}

\subsubsection{証明}\label{ux8a3cux660e}

1は明らか。 2は、次の定理から従う

\subsection{定理 0.2
(カントール・ベルンシュタイン(シュレーダー)の定理)}\label{ux5b9aux7406-0.2-ux30abux30f3ux30c8ux30fcux30ebux30d9ux30ebux30f3ux30b7ux30e5ux30bfux30a4ux30f3ux30b7ux30e5ux30ecux30fcux30c0ux30fcux306eux5b9aux7406}

AがBへの単射fとBからAへの単射g が存在する時、AからBへの全単射が存在する

\subsubsection{略証}\label{ux7565ux8a3c}

f: A -\textgreater{} B g: B -\textgreater{} A
\end{document}